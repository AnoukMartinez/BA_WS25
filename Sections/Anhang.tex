\clearpage
\section{Anhang}
\label{sec:appendix}

\subsection{Vorgehen bei der Untersuchung der Curricula}
Um auszuarbeiten, wie die aktuelle Situation an deutschen Hochschulen und Universitäten ist, wurde der Hochschulkompass der Hochschulrektorenkonferenz verwendet \cite{hochschulkompass}.
Auf der Webseite wurden Studiengänge der Informatik gesucht, und nach folgenden Kriterien sortiert:

\begin{itemize}
    \item Abschluss Bachelor/Bakkalaureus (Da Programmieranfänger betrachtet werden sollen, macht es im Kontext keinen Sinn, weiterführende Studiengänge oder Masterprogramme zu berücksichtigen)
    \item Studientyp Grundständig (Siehe Abschluss)
    \item Fachsuche Informatik (Speziellere Studiengänge wie Bioinformatik und Wirtschaftsinformatik wurden ausgeschlossen, um Überschneidungen an den Schulen zu meiden. Es wurde immer ein Studiengang pro Institution untersucht, der sich möglichst nah an der Allgemeinen Informatik kategorisieren lässt)
    \item Studienfeld Angewandte Informatik oder Informatik (Siehe Fachsuche)
    \item Studienformen Vollzeitstudium (Ein weiteres Kriterium, um Überschneidungen zu meiden und die Studiengänge weiter auszusortieren)
    \item ohne Lehramt (Wurde in den Filtern aussortiert, da die Lerninhalte sowohl Informatik als auch Erziehungswissenschaften umfasst, und somit nicht im Fokus der Arbeit liegen)
\end{itemize}

Nach der Anwendung der Filter wurden noch insgesamt 425 Treffer angezeigt, die weiterhin auf 121 Ergebnisse reduziert werden konnten. Folgende Ausschlusskriterien wurden verwendet, um Studiengänge auszusortieren. Wenn eins der Kriterien zutrifft, wird der Studiengang aussortiert.

\begin{itemize}
    \item Kein Allgemeiner Informatikstudiengang
    \item Zweitfach oder Nebenfach Informatik
    \item Teilzeitstudium
    \item Mehrere Studienorte für einen Studiengang (Hierbei sind die angebotenen Module teils nicht eindeutig zuordbar)
    \item Doppelte Studiengänge für eine Institution (Etwa wenn sowohl Angewandte, als auch Allgemeine Informatik angeboten wird. Hierbei wurde sich immer für den Allgemeineren Studiengang entschieden. Zwischen internationalen und deutschen Studiengängen wurde immer der deutsche Studiengang gewählt)
\end{itemize}

Aufgrund dessen, dass die Curricula Analyse nicht der Schwerpunkt der Bachelorarbeit sein sollte, musste abgewägt werden, wie viel Zeit in das Thema investiert werden soll.
Hierbei wurden mehrere Risiken gesehen. Zum einem ist es möglich, dass man durch Internetrecherche alleine nicht erschließen kann, welche Inhalte ein Modul hat. Zum anderen ist, das größere Risiko wahrscheinlich, der Zeitfaktor. Es ist ungewiss, wie lange es dauert, jedes Modul zu untersuchen, da jede Institution ihre Informationen anders sortiert, bereitsstellt und handhabt.

Für Methodiken, um die Menge an Daten in einem realistischen Zeitraum zu bearbeiten, wurden folgende in Erwägung gezogen.
\begin{itemize}
    \item Ein festgelegter Zeitraum, in dem die Daten betrachtet werden, und nur die, die in dem Zeitraum machbar sind.
    \item Ranking der Hochschulen nach Anzahl der Studierenden, und Betrachtung der 50 größten Insititutionen in Reihenfolge.
\end{itemize}

Es wurde notiert, dass das ungewisseste Kriterium hier die Zeit war, die benötigt wird, um einen Studiengang zu untersuchen. In einer isolierten Probe wurden fünf zufällige Universitäten betrachtet (In diesem Fall die ersten fünf Suchergebnisse des Hochschulkompasses, die HS Furtwangen, die Ruhruniversität Bochum, die Hochschule Fulda, die Friedrich-Schiller-Universität Jena, und die Hochschule Konstanz Technik, Wirtschaft und Gestaltung). Hierbei dauerte es etwa 10 Minuten, um entsprechende Informationen über alle 5 Studiengänge zu erlangen.
Bei den 121 verbleibenden Informatik Studiengängen wurde die Zeitdauer also grob auf 4 Stunden eingeschätzt, ein realistischer Zeitraum zur Sammlung der Daten.

\subsubsection{Untersuchung der Daten}
Bei der Analyse der Curricula wurde systematisch vorgegangen. Mithilfe eines Webscrapers wurden alle nötigen Informationen als JSON extrahiert, und in einer Excel Tabelle untersucht. So geschah auch die oben erwähnte Aussortierung der Studiengänge, die nicht auf die Kriterien zutreffen.
% Die Excel Tabelle mit den extrahierten Daten lässt sich im Repository der Bachelorarbeit finden \cite{repo_xlsx}.