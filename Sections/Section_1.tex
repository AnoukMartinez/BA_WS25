\clearpage
\section{Einleitung}
\label{sec:intro}

Obwohl sich seit Jahren bemüht wird, den Einstieg in die Programmierung einfacher zu gestalten, sind die Abbruchquoten in der Informatik in Deutschland im Verhältnis zu der Anzahl der Studierenden\footnote{In der Arbeit werden generell geschlechtsneutrale Begriffe verwendet, insofern sich diese nicht auf eine spezielle Person beziehen. Die verwendeten Personenbezeichnungen beziehen sich auf alle Geschlechter, und sollten allgemein zu verstehen sein.} weiterhin hoch \cite{destatis}. Dies lässt sich besonders im Vergleich zu anderen Studienfeldern erkennen \cite{dhzw}.
Es entstehen immer wieder ähnliche Probleme und Frustrationen, sowohl bei den Studierenden als auch beim Lehrpersonal \cite{mcdonald}.

Hierbei ergibt sich die Frage, ob das Lernen von Programmieren individuelle Unterschiede je nach Person hat, die diese Problematiken besonders begünstigen. In dieser Arbeit soll untersucht werden, welche möglichen Stärken und Schwächen es gibt, die sich deutlicher in verschiedenen Personen auszeichnen, und mit welchen Programmierparadigmen diese jeweiligen Lerntypen dabei unterstützt werden können, Programmierung leichter zu erlernen.
Speziell betrachtet hierbei wird das funktionale Paradigma. Es wurde sich für funktionale Programmierung entschieden, da das Paradigma ein nicht sehr weit verbreiteter Ansatz in den meisten Informatik-Einführungskursen ist (siehe \nameref{sec:curriculares}). Es stellt sich hierbei die Frage, ob die Schwierigkeiten der Studierenden beim Lernen von Programmierung nicht durch herkömmliche Paradigmen verstärkt werden, und ob ein unkonventioneller Ansatz hierbei einen Unterschied machen kann.

Folgende Forschungsfragen werden betrachtet:

\begin{enumerate}
    \item Fällt es jedem gleich leicht, Programmieren zu lernen, oder gibt es Hänge zu einem bestimmten Paradigma?
    \item Welche Vor- und Nachteile haben gängige Paradigmen für die jeweiligen Lerntypen?
    \item \textbf{Fokus:} Welchen Lerntypen würde funktionale Programmierung beim Lernen von Entwicklung helfen, und welchen Typen würde der Ansatz eher schaden?
\end{enumerate}
