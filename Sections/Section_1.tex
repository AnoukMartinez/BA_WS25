\clearpage
\section{Einleitung}
\label{sec:intro}

In vielen Studiengängen der Informatik wird Programmierung mit herkömmlichen Objektorientierten Paradigmen beigebracht. (TODO Quelle) Allerdings ergeben sich besonders in den Anfängen der Lehre häufig bekannte Probleme immer wieder, die bei vielen Studierenden zu Frustrationen und Verwirrung führen. (TODO Quelle) Es stellt sich die Frage, ob das Lernen von Programmieren nicht individuelle Unterschiede je nach Person hat. In dieser Arbeit soll untersucht werden, welche möglichen Stärken und Schwächen es gibt, die sich deutlicher in verschiedenen Personen auszeichnen, und wie diese jeweiligen Lerntypen dabei unterstützt werden können, Programmierung zu erlernen.
\\
Ein weiterer Fokuspunkt der Forschungsarbeit ist ebenfalls, wie verschiedene Programmierparadigmen diese Lerntypen unterschiedlich stützen können, und besonders ob funktionale Programmierung als eher unübliches Paradigma Studierende möglicherweise besonders gut unterstützen kann.
