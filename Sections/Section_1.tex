\clearpage
\section{Einleitung}
\label{sec:intro}

Obwohl sich seit Jahren bemüht wird, Programmierung einfacher für Studienanfänger zu gestalten, sind die Abbruchquoten in der Informatik dennoch vergleichsweise hoch \cite{dhzw}. 
Es ergeben sich immer wieder ähnliche Probleme und Frustrationen, sowohl auf Seiten der Studierenden, als auch der der Dozenten \cite{mcdonald}. 

Hierbei ergibt sich die Frage, ob das Lernen von Programmieren nicht individuelle Unterschiede je nach Person hat, die diese Problematiken besonders begünstigen. In dieser Arbeit soll untersucht werden, welche möglichen Stärken und Schwächen es gibt, die sich deutlicher in verschiedenen Personen auszeichnen, und mit welchen Programmierparadigmen diese jeweiligen Lerntypen dabei unterstützt werden können, Programmierung leichter zu erlernen.
Speziell betrachtet hierbei wird das funktionale Paradigma, ein nicht sehr weit verbreiteter Ansatz in den meisten Informatik Einführungskursen (siehe \nameref{sec:curriculares}).

Folgende Forschungsfragen werden in den Fokus gestellt:

\begin{enumerate}
    \item Fällt es jedem gleich leicht, Programmieren zu lernen, oder gibt es Hänge zu einem bestimmten Paradigma?
    \item Welche Vor- und Nachteile haben gängige Paradigmen für die jeweiligen Lerntypen?
    \item \textbf{Fokus:} Welchen Lerntypen würde funktionale Programmierung beim Lernen von Entwicklung helfen, und welchen Typen würde der Ansatz eher schaden?
\end{enumerate}
