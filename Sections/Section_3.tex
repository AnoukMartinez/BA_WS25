\clearpage
\section{Forschungsteil}
\label{sec:work}

% Im Grunde genommen sollte alles in diesem Abschnitt noch einmal umgeschrieben werden, das ist im Moment noch etwas chaotisch.
% Einleitung

Um die Frage zu beantworten, ob es bestimmte Hänge zu einem oder dem anderen Paradigma gibt, muss zuerst beantwortet werden, wie die Paradigmen mit den Lerntypen zusammenhängen.
Hierzu wird betrachtet, welche Lerntypen Vorteile beim Lernen der Computational Thinking (CT) Aspekte haben könnten.
Außerdem wird geprüft, welche Empfehlungen für die Lerntypen laut Felder und Silvermann mit den Ausrichtungen der Paradigmen zusammenhängen können. Es wird speziell Objektorientierung (OO) als häufigstes Paradigma in Einführungskursen, und funktionale Programmierung (FP) als Fokuspunkt der Arbeit betrachtet.

% Disclaimer
Es werden hierbei vermehrt Annahmen zu den Vorlieben der Lerntypen gemacht, aufgrund der Empfehlungen von Felder und Silvermann im Relation zu den Merkmalen der Paradigma. Aufgrund der zeitlichen Beschränkung der Arbeit können diese Annahmen nicht empirisch überprüft werden.

\subsection{Lerntypen im Zusammenhang mit Computational Thinking}

Nicht jeder Lerntyp hat klare Vor- und Nachteile beim Lernen der CT Aspekte. Es lassen sich allerdings einige klar definierte Hänge herausstellen.

% VISUELL/AUDITIV
Visuelle Lerner können einen Vorteil beim Erlernen von Abstraktion haben, da grafische Darstellungen eine sehr geeignete und beliebte Form sind, um das Konzept von Abstraktion zu erklären. % !!! KEIN QUELLE

% REFLEKTIV/AKTIV
Reflektiv besser beim Debugging, weil sie eher veranlagt sind, ihre Lösung zu hinterfragen.
Aktiv risikobereiter, einfach machen und schauen, passt nicht so ganz auf das Konzept von Debugging/Lösungen analysieren.

% SENSORISCH/INTUITIV
Algorithmisches Denken: Auch hier hat der intuitive Lerntyp eher einen Vorteil. % WIESO?
Abstraktion: Intuitiv explizit Vorteil weil der Typ besser in Abstraktion generell ist (Wird explizit im Handout gesagt)

% SEQUENTIELL GLOBAL
Dekomposition definitiv sehr gut für sequentielle Lerner. -> FP besser, weil Teilprobleme die nacheinander gelöst werden.
% \subsection{Lerntypen im Zusammenhang mit Computational Thinking}

\subsection{Eignung der Lerntypen für die Programmierparadigmen}
% VISUELL/AUDITIV
% Funktionale Programmierung Mathematische Abstraktion -> Evtl besser den Zusammenhang zu verstehen, wenn man die Ähnlichkeiten in den Formeln sehen kann? Ist aber sehr spekulativ

% REFLEXIV/AKTIV
Ein reflektiver Ansatz ist hier bei beiden Paradigmen empfehlenswert, da die praktische Umsetzung von Programmierkentnissen größtenteils in Einzelarbeit erfolgt.

% SENSORISCH/INTUITIV
Hierbei zeigt sich ein deutlicher Hang zu den Paradigmen. Sensorische Typen werden höchstwahrscheinlich OO bevorzugen. Die Klassen der OO modellieren Objekte der echten Welt, und haben somit einen einfach verständlichen Bezug zu echten Problemen.
Intuitive Typen hingegen sind besser darin, Abstraktion anzuwenden, und werden daher eher FP bevorzugen. Sie können aufgrund ihres Hanges zu Innovation und Verständnis von abstrakteren Konzepten ohne direkten Bezug zur echten Welt schneller Konzepte der FP verstehen und umsetzen. Die deklarative Art, wie FP aufgebaut ist bietet diesem Lerntypen einen eindeutigen Vorteil.

% SEQUENTIELL/GLOBAL
Schwierig I guess aber wahrscheinlich sequentiell besser FP weil step by step probleme lösen usw,

% ----------- %
% NOTIZEN
Größtest Problem momentan: Ok, man hat einen Zusammenhang zwischen den Lerntypen und CT, man hat den Zusammenhang zwischen den Lerntypen und den Paradigmen... Aber wie vereint man jetzt das ganze? Hat das überhaupt eine Relevanz zueinander? Kann man wirklich sagen dass Dekomposition/Algorithmen/Debugging/Abstraktion in FP wichtiger ist als in OO? (Eigentlich nicht, denke ich)
Wie also den Zusammenhang herstellen?

% ---------

% Zusammenfassung und explizit darstellen, welche Kombination von Lerntypen Aspekten die optimalen Bedingungen für FP ergibt