\clearpage
\section{Fazit}
\label{sec:conclusion}

Abschließend werden noch einmal alle Ergebnisse der Bachelorarbeit insgesamt betrachtet und in Beziehung zueinander gesetzt. Zudem erfolgt eine kurze Reflexion zum Prozess der Arbeit, sowie mögliche Erweiterungen der Forschungsfrage in zukünftigen Arbeiten.

\subsection{Schlussfolgerungen aus Theorie und Praxis}
Abschließend lässt sich also sagen, dass die Ergebnisse der Arbeit darauf hindeuten, dass ein Zusammenhang zwischen den verschiedenen Lerntypen und ihrer Eignung für funktionale Programmierung (FP) besteht.
Im Verlaufe der Arbeit wurden sowohl Vor- als auch Nachteile für die einzelnen Lerntypendimensionen identifiziert und mithilfe eines Selbstversuches analysiert. Die Erkenntnisse legen nahe, dass Probleme von Programmieranfangenden möglicherweise reduziert werden können, wenn die Stärken und Schwächen dieser berücksichtigt werden. Dies könnte zu einer geringeren Abbruchrate beitragen und den Einstieg in das Studium für neue Studierende generell einfacher gestalten.
Dies könnte beispielsweise so umgesetzt werden, dass zu Beginn des Studiums eine Selbsteinschätzung der Studierenden stattfindet, hinsichtlich ihrer eigenen Lerntypdimensionen. Somit können individuelle Stärken unterstützt werden. Je nachdem, mit welchem Ansatz sich eine Person mehr identifiziert, könnten anschließend abgestimmte Aufgaben für die jeweiligen Gruppen verteilt werden. Wenn Studierende zu Beginn des Studiums unterstützt werden können, liegt es nahe anzunehmen, dass sie mithilfe der Grundlagen auch im weiteren Verlaufe ihres Studiums weniger Schwierigkeiten haben werden.

Auch könnten Studierende selbst ohne zusätzliches Lehrmaterial besser verstehen, wie sie am besten die Inhalte der Vorlesungen nachvollziehen können. Gegebenenfalls erkennen sie durch ein Arbeiten mit einem speziellen Programmierparadigma wie FP, dass noch Schwierigkeiten in einigen Computational Thinking (CT) Aspekten bestehen, auf die sie sich außerhalb der Vorlesung konzentrieren können.

Im Hinblick auf die zu Beginn formulierten Forschungsfragen lassen sich folgende Schlussfolgerungen ziehen.

\begin{enumerate}
    \item \textbf{Fällt es jedem gleich leicht, Programmieren zu lernen, oder gibt es Hänge zu einem bestimmten Paradigma?} Abschließend lässt sich auf Basis der in der Arbeit erlangten Erkenntnisse sagen, dass Hänge zu Paradigmen erkannt wurden. Welche Paradigmen von wem bevorzugt werden, hängt von den individuellen Lerntyp-Präferenzen ab, könnte allerdings auch von der Gestaltung der Vorlesungen beeinflusst werden. Eine genauere empirische Überprüfung der Thesen steht noch aus.
    \item \textbf{Welche Vor- und Nachteile haben gängige Paradigmen für die jeweiligen Lerntypen?} Es wurden im Laufe der Arbeit eindeutige Vor- und Nachteile für FP und Objektorientierung (OO) gefunden. Andere Paradigmen wurden zunächst nicht betrachtet, könnten allerdings in der Zukunft auch in den Kreis der zu betrachtenden Paradigmen aufgenommen werden.
    \item \textbf{Welchen Lerntypen würde funktionale Programmierung beim Lernen von Entwicklung helfen, und welchen Typen würde der Ansatz eher schaden?} Im Forschungsteil der Arbeit konnte identifiziert werden, welche Kombination von Ausprägungen der Lerntypdimensionen ein Lernen von FP besonders begünstigt. Die Erkenntnisse stützen sich hierbei auf den in der Arbeit gezogenen Schlussfolgerungen.
\end{enumerate}

\subsection{Reflexion}
Dieser Abschnitt betrachtet die Umsetzung der Bachelorarbeit in Hinsicht auf die ursprünglich geplante Durchführung und wägt ab, wieso Schwierigkeiten auftraten, und welche Verbesserungen sinnvoll wären.

\subsubsection{Erfolge}
Einige Aspekte der Bachelorarbeit fielen leichter als zunächst angenommen. Besonders hat hierbei das Felder-Silverman Lerntyp Modell (FSLSM) geholfen. Dies konnte man sehr gut in den Kontext der Informatik setzen, unter anderem weil das Modell auch in der Vergangenheit in anderen Studien des Feldes als Richtlinie verwendet wurde.

Die Verfassenden des Modells legen zudem einen besonderen Wert darauf, dass das Modell und die Theorie weiterhin frei zugänglich bleiben, und stellen auf ihrer Webseite sehr viele Links und Ressourcen zur Verfügung. Die Ressourcen beinhalten unter anderem auch Belege zur Verlässlichkeit des Lerntypenmodell-Fragebogens, was die Entscheidung zur Verwendung des Modells als Mittel sehr einfach machte.

Der Fragebogen der offiziellen Seite war zunächst nicht funktional, aber auf Anfrage stellte sich heraus, dass dies daran lag, dass international versucht wurde, auf eine Webseite der University of Pennsylvania zuzugreifen. Mithilfe eines Virtual Private Networks (VPN) konnten diese Probleme behoben werden, und die Seite konnte normal verwendet werden.

Weiterhin war das Lernen mithilfe des Kurses der University of Pennsylvania sehr hilfreich. Der Kursinhalt war leicht verständlich und gut aufgebaut. Als Vorbereitung auf den praktischen Selbstversuch wurden einige Übungsaufgaben des Kurses bearbeitet. Dies half erheblich mit dem Erlernen der Haskell Syntax und der generellen Denkweise von FP.

\subsubsection{Schwierigkeiten}
Eines der frühesten Probleme der Bachelorarbeit fand sich im Rechercheteil. Zum Begriff CT ließ sich keine einheitliche Definition finden, und es war sehr schwer abzuwägen, welche Studien am vertrauenswürdigsten sind.
Um diesem entgegenzuwirken, wurden mehrere Studien und Literatur zu dem Thema betrachtet, die es sich vornehmen, die bisherigen Forschungsergebnisse aller Studien zusammenzufassen und auf einen einheitlichen Stand zu bringen. Um eine möglichst objektive Auswahl zu treffen, wurden vier Aspekte gewählt, die in der Forschung am häufigsten diskutiert und untersucht wurden. Hierzu wurden mögliche Unteraspekte unter größeren CT Aspekten gruppiert.

Zudem war eines der größten Probleme der Bachelorarbeit, einen Zusammenhang der drei Themenfelder, CT, FP und das FSLSM, zu finden. Es wurde ein klarer Zusammenhang zwischen CT und den Lerntypen, sowie FP und den Lerntypen gefunden. Diese Vor- und Nachteile dann allerdings wiederum in der dritten Ebene zu kombinieren erwies sich als besonders zeitaufwendig.
Es war sehr schwierig zu bestimmen, wie CT und FP zusammenhängen, und welche Aspekte eine größere oder kleinere Rolle in den einzelnen Paradigmen spielen. Jeder CT ist in seiner eigenen Weise für das Programmieren allgemein wichtig.
Da es keine vorhandene Forschung zum Zusammenhang von Programmierparadigmen und CT gab, basieren die Ergebnisse dieses Abschnitts größtenteils auf eigenen Schlussfolgerungen und Annahmen.
Es wurden verschiedene Überschriften für die Themen entworfen, die die Gedankenprozesse ein wenig eingrenzten. Letztlich wurden die Abschnitte zunächst wieder gelöscht und durch eine stichpunktartige Herangehensweise neu formuliert, um die Gedanken klarer zu formulieren. Zudem half es, eine visuelle Zusammenfassung der Schlussfolgerungen der einzelnen Abschnitte in Form der Tabellenabbildungen zu haben. Auf die Visualisierungen wurde auch immer wieder zurückgegriffen.
Letztendlich orientierte sich die Struktur stärker an den Forschungsfragen als an der ursprünglich geplanten Gliederung der Abschnitte.

Zusätzlich erwies sich als herausfordernd, die Wichtigkeit der einzelnen CT und FP Aspekte angemessen herauszustellen. Beispielsweise ist Rekursion ein zentrales Konzept in der FP, ließ sich im Forschungsteil aber nur mit algorithmischem Denken in Verbindung setzen. Es war schwierig angemessen zu artikulieren, welche Aspekte der FP besonders relevant sind, insbesondere durch den Mangel an nötiger praktischer Erfahrung im Thema. Dadurch ist es möglich, dass einige wichtige Schwerpunkte der FP nicht ausreichend im Forschungsteil berücksichtigt wurden.

Eine weitere Schwierigkeit war insgesamt, die Forschungsfragen mit dem Selbstversuch zu belegen. Der Zweck des Selbstversuches war es, einen Programmieranfangenden zu simulieren, der noch keine Erfahrungen in funktionaler Programmierung gemacht hat. Die Autorin hatte zwar keine funktionalen Programmierkenntnisse, allerdings bereits generelles Programmierwissen. Es bleibt daher unklar, inwiefern die bereits angeeigneten CT Aspekte in der Lösungsfindung geholfen haben. Insbesondere die Aspekte, in denen ein Vorteil gegeben war, also die Abstraktion und Dekomposition, sind Aspekte der Informatik, die im Verlaufe des Studiums immer wieder weitergebildet und trainiert werden.
Besonders bei der Implementierung fiel auf, dass eher auf herkömmliche prozedurale Denkstrukturen zurückgegriffen wurden, die anschließend in funktionale Funktionen umgewandelt wurden.
Stattdessen sollten die Programmierfähigkeiten von Grund auf mit funktionaler Programmierung betrachtet werden. Eine Diskussion dazu, wie dieses Problem umgangen werden könnte, findet sich im Abschnitt \nameref{sec:empirical}.

Zudem, als ein eher allgemeines Problem des Selbstversuches, war es sehr schwer und zeitaufwendig, sich mit der neuen Syntax von Haskell auseinanderzusetzen. Die Sprache ist fern genug von allen im Studium erlernten Programmierkenntnissen, um in der Bearbeitung eine Schwierigkeit darzustellen.
Das lokale Einrichten der Sprache, sowie die Verwendung der REPL war allerdings sehr einfach, und gab schnelles Feedback beim Durchführen der Aufgaben.

\subsection{Ausblick und Weiterführende Forschung}\label{sec:future}
Abschließend werden mögliche Weiterentwicklungen und offene Fragestellungen in Hinblick auf die Forschungsfragen betrachtet, die sich aus den Schlüssen der Arbeit ergeben.

\subsubsection{Unterstützung von benachteiligten Lerntypen}
Ein Aspekt, der in der Arbeit vernachlässigt wurde, ist die Nachteilausgleichung. Laut des FSLSM sollen die Lerntypen idealerweise nicht als Richtlinie für Lernmethoden verwendet werden. Allerdings ist es möglich, Personen mit schwächer ausgebildeten Dimensionen der Lerntypen zu unterstützen.
Beispielsweise sind herkömmliche Vorlesungen in Informatikstudiengängen eher in einer Art und Weise strukturiert, die verbalen Lerntypen einen Vorteil bieten würden. Personen mit schwächer ausgebildeten verbalen Kompetenzen könnten dabei unterstützt werden, diese Kompetenzen zu entwickeln. Parallel könnte diesen Personen allerdings auch gefördert werden, erste CT Kompetenzen zu erwerben, damit diese über einen längeren Zeitraum nicht den Anschluss an die Lerninhalte verlieren. Im Feld der Informatik wäre dies zum Beispiel durch die Verwendung der blockbasierten, visuellen Programmiersprache Scratch möglich.
Letztendlich geht es nicht darum, allen Lerntypen gerecht zu werden, sondern gemeinsame Lösungen zu finden, die sicherstellen, dass alle im gleichen Tempo lernen können.

Die Frage, wie dieser gemeinsame Grund für alle Programmiereinsteiger in Studiengängen der Informatik mithilfe der Lerntypen gefunden werden kann, könnte eine eigene Forschungsfrage für eine zukünftige Arbeit sein. Hierbei könnten verschiedene Ansätze zur Umstrukturierung einer "klassischen" Vorlesung entwickelt werden und gegebenenfalls unter Studienbeginnenden der Informatik vorgestellt werden.

\subsubsection{Erweiterung der Curricula-Analyse}
Aufgrund des Zeitmangels und weil die Curricula-Analyse nicht der Schwerpunkt der Bachelorarbeit war, wurden die Daten ausschließlich auf verwendete Programmiersprachen sowie Paradigmen untersucht. Was allerdings noch eine interessante Erweiterung wäre, könnte sein, die Kursinhalte selbst noch näher zu untersuchen. Dazu könnten entweder die Modulhandbücher herangezogen werden, die allerdings begrenzte Informationen über die Vorlesungen selbst zur Verfügung stellen. Andererseits könnten ausgewählte Universitäten und Hochschulen kontaktiert werden, um Zugang zu den detaillierten Lehrinhalten zu erlangen.
Beispielsweise könnten die wenigen Kurse, die sich mit funktionaler Programmierung beschäftigen, darauf untersucht werden, wie tief die Lehrinhalte hier wirklich gehen. Wird beispielsweise FP in der Praxis angewandt? Oder wird diese nur in der Theorie gelehrt und ist möglicherweise nur Teil einer kurzen Übersicht über alle verbreiteten Programmierparadigmen?

\subsubsection{Untersuchung anderer Paradigmen in Hinsicht auf CT Aspekte und Lerntypen}
Im Rahmen der Bachelorarbeit wurde speziell OO und FP im Kontext CT und Lerntypen betrachtet. Bei der Curricula Analyse ist allerdings aufgefallen, dass oft OO und Prozedurale Programmierung zusammen gelehrt werden. Hierbei stellt sich die Frage, welche Kompetenzen die jeweiligen Ansätze besonders gut vermitteln können.
Möglicherweise gibt es noch ausgeprägtere Hänge der bestimmten Lerntypdimensionen zu einem anderen Paradigma, welches im Rahmen der Forschungsarbeit nicht betrachtet wurde.

\subsubsection{Untersuchung weiterer CT Aspekte}
Aufgrund dessen, dass es keine einheitliche Definition von CT gibt, wurden in der Arbeit die vier häufigsten Aspekte betrachtet. Es wurden allerdings zudem auch weitere Aspekte herausgestellt, die ebenso relevant für die Anwendung von CT sind. Diese Aspekte wurden im Rahmen der Arbeit nicht betrachtet, könnten allerdings in Zukunft ein weiterer Fokuspunkt einer weiterführenden Forschungsfrage sein. Hierbei könnten die zusätzlichen CT Aspekte weiter kategorisiert und hierarchisch geordnet werden.

\subsubsection{Untersuchung der Vor- und Nachteile in funktionaler Programmierung}\label{sec:empirical}
Aufgrund der zeitlichen Begrenzung wurde der praktische Versuch zur Untersuchung der Vor- und Nachteile der Lerntypen in funktionaler Programmierung mit der Autorin selbst durchgeführt. Da allerdings bereits vorher bekannt war, welche Lerndimensionen eher ausgeprägt sind, gab es einen Bias in der Untersuchung der Schwierigkeiten und Vorteile.
Um eine verlässlicherere Untersuchung durchzuführen, wäre es sinnvoll, den Versuch erneut anhand fremder Versuchspersonen durchzuführen, die sich im ersten Semester eines Informatikstudiums befinden.
Hierzu könnte ein sehr kurzer Einleitungskurs zu FP entworfen werden, anhand dessen die Probanden eine Aufgabe durchführen. Im Anschluss an den Programmierteil wird dann der Lerntyp der einzelnen Personen mittels des Felder Silverman Fragebogens festgestellt.

Außerdem ist es möglich, dass das betrachtete Problem im praktischen Versuch letztendlich doch etwas kurz war, um alle CT Aspekte zu betrachten. Beispielsweise war es schwierig, das Problem in Teilverantwortungen im herkömmlichen Sinne zu unterteilen. Auch die sequenziell/globale Dimension konnte aufgrund dieses Aspektes nicht viele Erkenntnisse gewinnen.
Eine Aufgabe wie der Sudoku Solver hätte letztendlich den zeitlichen Rahmen der Arbeit gesprengt, wäre allerdings zur genaueren Untersuchung der Vor- und Nachteile geeigneter gewesen. Auch in der Diskussion war kurzzeitig, statt einer kleineren Aufgabe vier Aufgaben zu bearbeiten, die sich jeweils auf einen CT Aspekt fokussieren. Dies wäre allerdings ebenfalls nicht mit dem gegebenen Zeitrahmen vereinbar gewesen, könnte allerdings eine gute Methodik darstellen, um die Forschungsfrage in Zukunft weiter zu untersuchen.

\subsubsection{Weitere Forschung zur Ausprägung der CT Aspekte in den Paradigmen}
Aufgrund dessen, dass im Rahmen der Bachelorarbeit wenige Quellen zum Thema CT im Zusammenhang mit Programmierparadigmen gefunden wurden, basieren die meisten Schlussfolgerungen des Forschungsteils auf eigenen Annahmen. Allerdings könnten die Ausprägungen der einzelnen CT Aspekte ebenfalls in der Praxis untersucht werden.
Durch die Stärken und Schwächen der einzelnen Lerntypen würden sich in einem praktischen Versuch möglicherweise Rückschlüsse darauf ziehen lassen, wie sich die Paradigmen auf einzelne CT Aspekte fokussieren.
Diese Art von Forschungsfrage könnte sich ebenfalls mit der Frage auseinandersetzen, wie man Programmieranfangende am besten in ihrem Lernprozess unterstützen kann.
