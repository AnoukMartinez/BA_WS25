\clearpage
\section{Fazit}
\label{sec:conclusion}

Abschließend werden noch einmal alle Ergebnisse der Bachelorarbeit gesammelt betrachtet, und in Beziehung zueinander gesetzt. Zudem erfolgt eine kurze Reflektion zum Prozess der Arbeit, sowie mögliche Erweiterungen der Forschungsfrage in zukünftigen Arbeiten.

\subsection{Schlussfolgerungen aus Theorie und Praxis}
Verbindung des Theorie- und Praxisteiles in Hinsicht auf die Forschungsfrage

Abschließend lässt sich also sagen, dass ein Zusammenhang der verschiedenen Lerntypen zu Programmierparadigmen bestehen könnte. Im Verlauf der Arbeit wurden 

\subsection{Reflexion}
Dieser Abschnitt betrachtet die Durchführung der Bachelorarbeit in Hinsicht auf die ursprünglich geplante Durchführung, und wägt ab wieso Schwierigkeiten auftraten.

\subsubsection{Erfolge}
Einige Aspekte der Bachelorarbeit fielen leichter als zunächst angenommen. Besonders hat hierbei das Felder-Silverman Lerntyp Modell (FSLSM) geholfen. Dies konnte man sehr gut in den Kontext der Informatik setzen, unter Anderem weil das Modell auch in der Vergangenheit in anderen Studien des Feldes als Richtlinie verwendet wurde.

Die Autoren des Modelles setzen zudem einen besonderen Fokus darauf, dass das Modell und die Theorie weiterhin profitfrei angeboten werden können, und stellen auf ihrer Webseite sehr viele Links und Ressourcen zur Verfügung. Die Ressourcen beinhalten unter anderem auch Belege zu der Verlässlichkeit des Lerntypenmodell Fragebogens, was die Entscheidung zur Verwendung des Modells als Mittel sehr einfach machte.

Der Fragebogen der offiziellen Seite war zunächst nicht funktional, aber auf Anfrage stellte sich heraus dass dies an dem Zugriffsort lag. Mithilfe einer VPN Verbindung wurde die Seite von einem amerikanischen Server aus abgerufen, und konnte somit normal verwendet werden. % Faktisch korrekt?

\subsubsection{Schwierigkeiten}
Eines der frühesten Probleme der Bachelorarbeit fand sich im Rechercheteil. Zum Begriff Computational Thinking ließ sich keine einheitliche Definition finden, und es war sehr schwer abzuwägen, welche Studien am vertrauenswürdigsten sind.
Um diesem entgegenzuwirken, wurden mehrere Studien und Literatur zu dem Thema betrachtet, die es sich vorhehmen, die bisherigen Forschungsergebnisse aller Studien zusammenzufassen und auf einen einheitlichen Stand zu bringen. Abhängig davon, wie weit verbreitet der Konsensus für einen CT Aspekt war, wurden vier Aspekte ausgewählt, die am meisten von den verschiedenen Forscher betrachtet wurden.

Zudem war eines der größten Probleme der Bachelorarbeit, einen Zusammenhang der drei Themenfelder, CT, FP und das FSLSM, zu finden. Es wurde ein klarer Zusammenhang zwischen CT und den Lerntypen, sowie FP und den Lerntypen gefunden, diese Vor- und Nachteile dann allerdings wiederum in der dritten Ebene zu kombinieren war letztendlich sehr zeitaufwendig.
Es war sehr schwierig zu bestimmen, wie CT und FP zusammenhängen, und welche Aspekte eine größere oder kleinere Rolle in den einzelnen Paradigma spielen. Jeder CT ist in seiner eigenen Weise für das Programmieren allgemein wichtig.
Es wurde zudem keine Forschung zum Thema Programmierparadigmen im Zusammenhang mit CT gefunden, weshalb die meisten Ergebnisse des Abschnittes auf eigenen Schlussfolgerungen und Annahmen beruhen.

Es wurde verschiedene Überschriften für die Themen entworfen, die die Gedankenprozesse ein wenig eingrenzten. Das Vorgehen war schlussendlich also, die einzelnen Abschnitte wieder zu löschen, und einfach Stichpunktartig zu schreiben, bis die Gedanken einigermaßen auf Papier gebracht waren. Zudem half es, eine visuelle Zusammenfassung der Schlussfolgerungen der einzelnen Abschnitte in Form der Tabellenabbildungen zu haben. Auf die Visualisierungen wurde auch immer wieder zurückgegriffen. Es wurde letztendlich eher nach den Forschungsfragen gearbeitet, als nach der ursprünglich geplanten Struktur der Abschnitte.

Eine weitere Schwierigkeit war insgesamt, mit dem praktischen Teil die Forschungsfrage zu untersuchen. Der Zweck des Praxisteiles war es, einen Programmieranfänger zu simulieren, der noch keine Erfahrungen in funktionaler Programmierung gemacht hat. Der Autor hatte zwar keine funktionalen Programmierkenntnisse, allerdings bereits generelles Programmierwissen. Es ist daher fraglich, inwiefern die bereits angeeigneten CT Aspekte in der Lösungsfindung geholfen haben. Insbesondere die Aspekte, in denen ein Vorteil gegeben war, also die Abstraktion und Dekomposition, sind Aspekte der Informatik, die im Verlaufe des Studiums immer wieder weitergebildet und trainiert werden.
Besonders bei der Implementierung fiel auf, dass immer wieder auf herkömmliche prozedurale Denkstrukturen zurückgegriffen wurden, die anschließend in funktionale Funktionen umgewandelt wurden . Dies ist beispielweise zu sehen in der Findung einer Lösung, die die Lösungsschritte in einer Liste dokumentiert und immer weiter erweitert, sprich also mit veränderlichen Werten arbeitet.
Stattdessen sollten die Programmierfähigkeiten von Grund auf mit funktionaler Programmierung betrachtet werden. Wie dieses Problem umgangen werden könnte, findet sich unter Anderem im Abschnitt \nameref{sec:future}.

Zudem, als ein eher allgemeines Problem des Praxisteiles, war es sehr schwer und zeitaufwendig, sich mit der neuen Syntax von Haskell auseinanderzusetzen. Die Sprache an sich ist fern genug von allen im Studium erlernten Programmierkenntnissen, um in der Bearbeitung ein Problem darzustellen.
Das lokale einrichten der Sprache, sowie die Verwendung der REPL war allerdings sehr einfach.

\subsection{Erweiterung der Forschungsfrage}\label{sec:future}
\subsubsection{Unterstützung von benachteiligten Lerntypen}
Ein Aspekt der in der Arbeit vernachlässigt wurde, ist der der Nachteilsausgleichung. Laut des FSLSM sollen die Lerntypen idealerweise nicht als Richtlinie für Lernmethoden verwendet werden. Allerdings ist es möglich, Personen mit schwächer ausgebildeten Dimensionen der Lerntypen zu unterstützen.
Beispielsweise sind herkömmliche Vorlesungen in Informatikstudiengängen eher in einer Art und Weise strukturiert, die verbale Lerntypen einen Vorteil bieten würde. Personen mit schwächer ausgebildeten verbalen Kompetenzen könnten dabei unterstützt werden, diese Kompetenzen zu entwickeln. Parallel könnte diesen Personen allerdings auch gefördert werden, erste CT Kompetenzen zu erwerben, damit diese über einen längeren Zeitraum nicht den Anschluss an die Lerninhalte zu verlieren. Im Feld der Informatik wäre dies zum Beispiel durch die Verwendung der blockbasierten, visuellen Programmiersprache Scratch möglich.
Letztendlich geht es nicht darum, allen Lerntypen gerecht zu werden, sondern gemeinsame Lösungen zu finden, die sicherstellen, dass alle im gleichen Tempo lernen können.

Die Frage, wie dieser gemeinsame Grund für alle Programmiereinsteiger in Studiengängen der Informatik mithilfe der Lerntypen gefunden werden kann, könnte eine eigene Forschungsfrage für eine zukünftige Arbeit sein. Hierbei könnten verschiedene Ansätze zur Umstrukturierung einer "klassischen" Vorlesung entwickelt werden, und gegenebenfalls unter Studienanfängern der Informatik vorgestellt werden.

\subsubsection{Erweiterung der Curricula Analyse}
Aufgrund des Zeitmangels, und weil die Curricula Analyse nicht der Schwerpunkt der Bachelorarbeit war, wurden die Daten ausschließlich auf verwendete Programmiersprachen, sowie Paradigma untersucht. Was allerdings noch eine interessante Erweiterung wäre, könnte sein, die Kursinhalte selbst noch näher zu untersuchen. Dazu könnten entweder die Modulhandbücher herangezogen werden, die allerdings begrenzte Informationen über die Vorlesungen selbst zur Verfügung stellen. Andererseits könnten ausgewählte Universitäten und Hochschulen kontaktiert werden, um Zugang zu den detaillierten Lehrinhalten zu erlangen.
Beispielsweise könnten die wenigen Kurse, die sich mit funktionaler Programmierung beschäftigen darauf untersucht werden, wie tief die Lehrinhalte hier wirklich gehen. Wird beispielsweise funktionale Programmierung in der Praxis angewandt? Oder wird diese nur in der Theorie gelehrt, und ist möglicherweise nur Teil einer kurzen Übersicht über alle verbreiteten Programmierparadigmen?

\subsubsection{Untersuchung anderer Paradigma in Hinsicht auf CT Aspekte und Lerntypen}
Im Rahmen der Bachelorarbeit wurde speziell Objektorientierte und funktionale Programmierung im Kontext CT und Lerntypen betrachtet. Bei der Curricula Analyse ist allerdings aufgefallen, dass oft OO und Prozedurale Programmierung zusammen gelehrt werden. Hierbei stellt sich die Frage, welche Kompetenzen die jeweiligen Ansätze besonders gut vermitteln können.
Möglicherweise gibt es noch ausgeprägtere Hänge der bestimmten Lerntypdimensionen zu einem anderen Paradigma, welches im Rahmen der Forschungsarbeit nicht betrachtet wurde.

\subsubsection{Untersuchung der Vor- und Nachteile in funktionaler Programmierung}
Aufgrund der zeitlichen Begrenzung wurde der praktische Versuch zur Untersuchung der Vor- und Nachteile der Lerntypen in funktionaler Programmierung mit dem Autor selbst durchgeführt. Da allerdings bereits vorher bekannt war, welche Lerntypen dieser besitzt, gab es einen gewissen Bias in der Untersuchung der Schwierigkeiten und Vorteile.
Um einen verlässlicheren Versuch durchzuführen, wurde es sinnvoll sein, den Versuch erneut anhand fremder Versuchspersonen durchzuführen, die sich im ersten Semester eines Informatikstudiums befinden.
Hierzu könnte ein sehr kurzer Einleitungskurs zu FP entworfen werden, anhand dessen die Probanten eine Aufgabe durchführen. Im Anschluss an den Programmierteil wird dann der Lerntyp der einzelnen Personen mittels des Felder Silvermann Fragebogens festgestellt.

\subsubsection{Weitere Forschung zur Ausprägung der CT Aspekte in den Paradigmen}
% Das ist noch ein bisschen WIP
Aufgrund dessen, dass im Rahmen der Bachelorarbeit wenige Quellen zum Thema CT im Zusammenhang mit Programmierparadigmen gefunden wurde, basieren die meisten Schlussfolgerungen des Forschungsteiles auf eigenen Annahmen.
Auch hier könnte detailliertere Forschungsarbeit geleistet werden, die sich nur auf diesen Teil der Arbeit fokussiert. Möglicherweise könnten die Ausprägungen der einzelnen CT Aspekte ebenfalls in der Praxis untersucht werden.
