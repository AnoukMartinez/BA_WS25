\clearpage
\section{Praxis}
\label{sec:practice}

Kurze Einleitung in den Sinn und Hintergrund des Praxisteiles

\subsection{Lerntypenanalyse}
Um das Selbstexperiment in Verbindung mit den Vor- und Nachteilen der Lerntypen in den verschiedenen Feldern setzen zu können, musste zunächst eine eigene Lerntypen Analyse durchgeführt werden.
Hierzu wurde der "Index of Learning Styles Questionnaire" Fragebogen verwendet \cite{ils_questionnaire}, ein Online Tool welches insgesamt 44 Fragen stellt, und dann auf Basis der Antworten sowohl die Lerntypen eines Individuums einschätzen kann. Die Lerntypen, die als Gegensatzpaare definiert sind, befinden sich auf einer 2 Dimensionalen Skala, die einschätzen soll, wie ausgeprägt der Hang zu einem Aspekt ist. Verwendet wurde hierbei die PDF Version des Fragebogens. Ein Wert von 1 bis 3 entspricht einer ausgewogenen Balance beider Aspekte der Dimension. Ein Wert von 5 bis 7 entspricht einem mäßigem Hand zu einem Aspekt. Ein Wert von 9 bis 11 entspricht einer starken Präferenz eines Lernstiles in der Dimension.
Die Verlässlichkeit dieses Fragebogens wurde mehrmals, sowohl von den ursprünglichen Autoren des Lernmodelles \cite{felder2005}, als auch von anderen Quellen \cite{zywno}, geprüft und bestätigt.

\subsubsection{Ergebnisse der persönlichen Evaluierung}
Die Fragen des ILS Fragebogens wurden in einer zufälligen Reihenfolge beantwortet, um zu vermeiden, dass bereits vorab klar ist, welche Frage welcher Typendimension korrespondiert. Dies ist möglich, da der Test so aufgebaut ist, dass jeweils ein Frageblock von 4 Fragen jede Dimension behandelt. Die erste Frage beispielsweise geht in die Wertung der Aktiv/Reflektiven Dimension ein, die 5 Frage beschäftigt sich wieder mit der Dimension, und die 9 Frage ebenfalls.

Die Fragen des ILS-Fragebogens wurden in zufälliger Reihenfolge beantwortet, um zu verhindern, dass bereits im Voraus offensichtlich ist, welche Frage welcher Dimension zugeordnet ist.
Dies ist möglich, da der Test so konzipiert ist, dass jede Dimension in Blöcken von vier Fragen abgedeckt wird. Beispielsweise fließt die erste Frage in die Bewertung der Aktiv/Reflexiv-Dimension ein, die fünfte Frage ebenfalls und ebenso die neunte.

Die Ergebnisse des Fragebogens lauten wie folgt.

\begin{figure}[h!]
    \centering
    \newcolumntype{C}{>{\centering\arraybackslash}X}
\begin{tabularx}{\textwidth}{l C C C C C C C C C C C C l}
    ACT & 11a & 9a & 7a & 5a & 3a & 1a & 1b & \colorbox{BurntOrange}{\textbf{3b}} & 5b & 7b & 9b & 11b & REF \\
    SEN & 11a & \colorbox{BurntOrange}{\textbf{9a}} & 7a & 5a & 3a & 1a & 1b & 3b & 5b & 7b & 9b & 11b & INT \\
    VIS & 11a & 9a & \colorbox{BurntOrange}{\textbf{7a}} & 5a & 3a & 1a & 1b & 3b & 5b & 7b & 9b & 11b & VRB \\
    SEQ & 11a & 9a & 7a & 5a & 3a & 1a & \colorbox{BurntOrange}{\textbf{1b}} & 3b & 5b & 7b & 9b & 11b & GLO \\
\end{tabularx}
    \caption{Ergebnisse des ILS Fragebogens, visualisiert}
\end{figure}

Demnach lässt sich schließen dass der Autor Hänge zu folgenden Dimensionen aufweist.

\begin{itemize}
    \item Ein ausgewogenes Verhältnis der Aktiven/Reflektiven Dimension
    \item Ein starker Hang zum sensorischem Typen 
    \item Ein mäßiger Hang zum visuellen Typen
    \item Ein ausgewogenes Verhältnis der Sequentiellen/Globalen Dimension
\end{itemize}

% Problem as I am writing this ist Section 3 noch nicht so ganz fertig und das hier könnte daher noch anders ausfallen

Die Hänge der einzelnen Lerntypen wird im praktischen Versuch anschließend in Bezugzu den Vor- und Nachteilen für das Erlernen und Anwenden der Computational Thinking Aspekte und der verschiedenen Paradigmen gesetzt.
Theoretisch sollte es in folgenden Bereichen Vor- und Nachteile geben.

\begin{itemize}
    \item Ein Vorteil beim Anwenden von Abstraktion
    \item Ein Vorteil im Aspekt des Debuggens, und der Analyse der gefundenen Lösungen
    \item Ein möglicher Nachteil beim algorithmischen Denken insgesamt
    \item Kein klarer Vor- oder Nachteil in der Dekomposition, aufgrund eines ausgewogenen Verhältnisses in der sequentiell/globalen Dimension
\end{itemize}

\subsection{Problembeschreibung}